\documentclass{article}

\usepackage[utf8]{inputenc}	%ces deux premiers packages
\usepackage[T1]{fontenc}		%concernent le formattage
\usepackage[french]{babel}	%Langue Français
\usepackage{hyperref}
\usepackage{listings}

\lstdefinestyle{customc}{
    belowcaptionskip=1\baselineskip,
    breaklines=true,
    frame=L,
    xleftmargin=\parindent,
    language=C,
    showstringspaces=false,
    basicstyle=\footnotesize\ttfamily,
    keywordstyle=\bfseries\color{green!40!black},
    commentstyle=\itshape\color{purple!40!black},
    identifierstyle=\color{blue},
    stringstyle=\color{orange},
}

\lstdefinestyle{customasm}{
    belowcaptionskip=1\baselineskip,
    frame=L,
    xleftmargin=\parindent,
    language=[x86masm]Assembler,
    basicstyle=\footnotesize\ttfamily,
    commentstyle=\itshape\color{purple!40!black},
}

\hypersetup{
    colorlinks=true,
    linkcolor=blue,
    filecolor=magenta,
    urlcolor=cyan,
    pdftitle={Overleaf Example},
    pdfpagemode=FullScreen,
}
\usepackage{mathtools, amsfonts, amsthm}
%pour écrire des maths
\usepackage{geometry}[a4paper,textwidth=160mm,textheight=250mm]
%pour gérer les marges et les espaces
\usepackage{xcolor}			%pour afficher des parties en couleur

\usepackage{graphicx}

\numberwithin{equation}{section}    % numérotation des eqns selon la section

% Titre

\title{Documentation du Stellar Engine}
\author{Adam Ellouze}

\begin{document}				%Début du corps du document

    \maketitle

    \section{Stellar Engine}
    Le Stellar Engine est un moteur de jeu conçu pour le jeu 'Super Stars Align', un jeu Platformer inspiré de \href{https://en.wikipedia.org/wiki/Super_Mario_64}{Super Mario 64}. \vspace{1cm}

    \section{Technologie}
    Voici une liste détaillée des différentes technologies utilisées dans le Stellar Engine :

    \begin{itemize}
        \item \href{https://ldtk.io/}{LDtk}, un éditeur de niveaux.
        \item  \label{maped} \href{https://github.com/baylej/tmx/}{TMX}, un importeur de cartes \href{https://www.mapeditor.org/}{Tiled}.
        \label{sdl} \item \href{https://libsdl.com}{SDL}, une librairie de développement C.
        \item \label{toml}  \href{https://toml.io/en/}{TOML}, un langage de configuration.
    \end{itemize}

    \section{Configuration}
    Le Stellar Engine est (fonctionnalité à venir) configurable à l'aide du langage TOML(\ref{toml}). Voici un exemple de pseudocode :


    \begin{lstlisting}
    [window]
    width = 800
    height = 600
    fullscreen = false
    title = "Stellar Game"
    \end{lstlisting}

    \section{Développement tiers}
    Le développeur tiers - qui utilisera le moteur - devra inclure \textit{\textbf{stellar.c}}  et \textit{\textbf{stellar.h}}, qui comporteront les fonctions suivantes, contituant l'\textit{API} du moteur :

    \begin{lstlisting}
    void stellarCleanup() // Pour nettoyer le moteur
    void stellarConfig("config.toml") // Pour charger une configuration
    stellarPlayer("player.png") // Pour charger un SPRITE de joueur
    \end{lstlisting}

    \lstinputlisting[caption=Scheduler, style=customc]

    Il devra aussi utiliser les fonctions présentent dans \textit{\textbf{game.c}} pour charger la carte. Des exemples de codes pour l'\textit{I/O} - de l'Anglais Input/Output, Entrées/Sorties des manettes, clavier, souris, etc.. -  seront fournis dans l'optique de simplifier le développement de jeu à l'aide du \textbf{Stellar Engine}.\vspace{0.5cm}


    Le développeurs tiers devra programmer une carte avec les éditeurs de carte (\ref{maped}). [...] Voici une liste de choses à faire lors du développement d'un jeu utilisant le moteur : \vspace{0.5cm}

    \begin{enumerate}
        \item Programmer une carte
        \item Charger la carte
        \item Charger les SPRITES de joueur
        \item Implémenter les I/O
        \item Implémenter le jeu
    \end{enumerate}

    \subsection{Structure}
    La structure du moteur est la suivante : \vspace{0.5cm}

    \begin{itemize}
        \item src/stellar.c
        \item src/stellar.h
        \item src/include/
        \item CMakeLists.txt
        \item game.c
    \end{itemize} \vspace{0.5cm}

    L'essentiel est dans le dossier \textbf{\textit{src}}, qui comporte chaque élément de la librairie \textbf{TMX} - qui permet de charger les cartes exportées - dans le sous-dossier \textbf{\textit{include}}, et les fichiers à inclure (\textbf{\textit{stellar.c, stellar.h}}.

    \section{Compilation}
    Pour compiler le moteur depuis sa source, il faut avoir \ref{sdl} installé. Voici les étapes de compilation : \vspace{0.5cm}

    \begin{lstlisting}
        git clone https://github.com/Volt-Soft/Stellar-Engine.git
        cd Stellar-Engine
        mkdir build
        cd build
        cmake ..
        make
    \end{lstlisting} \vspace{0.5cm}

    Cela compilera pour Linux et Unix. Il y aura trois fichiers : \textbf{\textit{StellarDoc.pdf}}, \textbf{\textit{StellarEngine.a}} et \textbf{\textit{StellarGame}}.


    \section{License}
    Le moteur est distribué sous la licence Zlib, qui est une licence permissive permettant une utilisation libre et flexible.

    \begin{lstlisting}
        Copyright (c) 2024 Adam Ellouze <elzadam11@gmail.com>

This software is provided 'as-is', without any express or implied
warranty. In no event will the authors be held liable for any damages
arising from the use of this software.

Permission is granted to anyone to use this software for any purpose,
including commercial applications, and to alter it and redistribute it
freely, subject to the following restrictions:

1. The origin of this software must not be misrepresented; you must not
   claim that you wrote the original software. If you use this software
   in a product, an acknowledgment in the product documentation would be
   appreciated but is not required.
2. Altered source versions must be plainly marked as such, and must not be
   misrepresented as being the original software.
3. This notice may not be removed or altered from any source distribution.
    \end{lstlisting}

\end{document}				      %Fin  